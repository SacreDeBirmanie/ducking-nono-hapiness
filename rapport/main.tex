\documentclass[a4paper]{article}
\usepackage[utf8]{inputenc}
\usepackage[frenchb]{babel}
\usepackage{ifpdf}
\usepackage{hyperref}
\title{Processeu \textsc{Nono} 1et 2}
\author{Sébastien \textsc{Vallée}, Benjamin \textsc{Sientzoff}}
\date{\today}
\ifpdf
\hypersetup{
    pdfauthor={Sébastion Vallée, Benjamin Sientzoff},
    pdftitle={Réalisation processeur Nono-1 et Nono-2},
}
\fi
\begin{document}
	% page de garde avec sommaire
	\maketitle
	\vspace{5cm}
	\tableofcontents
	\newpage % passer à la page suivante
	
	\section*{Introduction}
	% présenter le cadre du projet
	% présenter le prjet en lui-même, idées de base, inspiration
	\paragraph{}{le paragraphe d'intro}
	
	\section{Implémentation}
	
		% présentation des principales classes, méthodes
		% à quoi elles servent?
		\paragraph{titre du paragraphe}{contenu}
		\paragraph{}{paragraphe sans titre}
		
		\subsection{Premier pattern}
		% l'observeur
		\paragraph{titre du paragraphe}{contenu}
		\paragraph{}{paragraphe sans titre}
		
	\section{Utilisation Nono-1}
		\subsection{Second pattern}
		% le décorateur
		\paragraph{titre du paragraphe}{contenu}
		\paragraph{}{paragraphe sans titre}
		
		\subsection{Troisième pattern}
		% toto
		\paragraph{titre du paragraphe}{contenu}
		\paragraph{}{paragraphe sans titre}
	
	
	\section*{Conclusion}
		\paragraph{}{je conclu}
		
\end{document}
